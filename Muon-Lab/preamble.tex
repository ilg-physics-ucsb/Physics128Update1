\RequirePackage{amsmath,amsfonts,amsthm,environ,mathtools,wasysym,tensor} % Math packages
\RequirePackage{bbm,bm}
\RequirePackage{lipsum,caption,graphicx} 

\RequirePackage{xcolor,xargs,wrapfig,setspace} %Formatting
\RequirePackage{hyperref}
\RequirePackage[most]{tcolorbox}
% Set font and line spacing
\renewcommand{\baselinestretch}{1.3}
\renewcommand{\familydefault}{\sfdefault}
 
% Define Colors for box classes
\definecolor{niceteal}{RGB}{120, 171, 145}
\definecolor{physicsconcept}{RGB}{161, 113, 26}
\definecolor{instrumentconcept}{RGB}{93, 78, 99}
\definecolor{analysisconcept}{RGB}{101, 112, 92}

% Default Box
\tcbset{colframe=niceteal,colbacktitle=niceteal,colback=white, arc=1mm,width=\textwidth,fonttitle=\sffamily\bfseries\large,halign title=flush center,tcbox width=auto limited,center,enlarge left by = 0 in}

%Custom Boxes -- I used enviroments because they are mor consitstent than tcolorbox built ins.
\newtcolorbox{QQP}[1][]{
enhanced, #1, width= \textwidth, colframe=physicsconcept, center,valign=center,
before skip=5mm, after skip=5mm,
leftrule=8mm, sharp corners=northwest,parbox=true,
enlarge left by = 0 in,
halign=flush left ,  fontupper= \small \setstretch{1.0}, title= Q,
detach title,
overlay={
        \node[ minimum width=\textwidth, anchor=north,xshift=4 mm,yshift=3mm] at (frame.west) {\tcbtitle};
        }
     }
     
\newtcolorbox{QEQ}[1][]{
enhanced, #1, width= \textwidth, colframe=niceteal, center,valign=top,
%before skip=5mm, after skip=5mm,
halign=center
     }
\environbodyname\BBODY



\newenvironmentx{wblurb}[5][1=r,2=0.4,3=title,4]
  {\wrapfigure[#5]{#1}{#2\textwidth}\tcolorbox[width=#2\textwidth, title=\normalsize \setstretch{1.0} #3,fontupper= \footnotesize \setstretch{1.0},#4]}
  {\endtcolorbox\endwrapfigure}
%%%%%%%%%%%%%%%%%%%%%%%%%%%%%%%%%%%%%%%%%%%%%%%%%%%%%%%%%%%%%%%%%%%%%%%%%%%%%%%%%%%%
%% Commands for Convenience
%%%%%%%%%%%%%%%%%%%%%%%%%%%%%%%%%%%%%%%%%%%%%%%%%%%%%%%%%%%%%%%%%%%%%%%%%%%%%%%%%%%%

%% Math Environments
\makeatletter
\newcommand{\oset}[3][0ex]{%
	\mathrel{\mathop{#3}\limits^{
			\vbox to#1{\kern-2\ex@
				\hbox{$\scriptstyle#2$}\vss}}}}
\makeatother

\NewEnviron{align**}
{\par\setstretch{1.0}\begin{align}\begin{split} 
\BBODY 
\end{split}\end{align}}

\NewEnviron{gather**}
{\begin{gather}\begin{split} 
\BBODY 
\end{split}\end{gather}}
%{\begin{gather}\begin{split}}{\end{split}\end{gather}}



%%Shortcuts
\newcommand{\llangle}{\l\langle}
\newcommand{\rrangle}{\r\rangle}
\renewcommand{\l}{\left}
\renewcommand{\r}{\right}

\renewcommand{\b}{\textbf}
\newcommand{\+}[1]{\,\text{#1}}

\newcommand{\bsm}{\boldsymbol}
\newcommand{\vphi}{\varphi}

%%Symbols
\newcommand{\dg}{^{\dag}}
\newcommand{\da}{\downarrow}
\newcommand{\ua}{\uparrow}
\renewcommand{\d}{\partial}
\newcommand{\del}{ {\oset{{_{_{_{_ {\rightharpoonup}}}}}}{\! \nabla }}\! }
\renewcommand{\.}{\cdot}
\renewcommand{\'}[1]{\ensuremath{ \vec{{#1}}\,}}
\renewcommand{\^}{\hat}
\newcommand{\dx}[2]{\frac{\text{d}\,{#1}}{(2\pi)^{#2}} \;}
\newcommand{\qeq}{\overset{!}{=}}
\renewcommand{\*}{^{\,*\!}}
\newcommand{\RA}{\rightarrow}
\newcommand{\T}{^{_T}}



%%Greek 
\newcommand{\w}{\omega}
\newcommand{\wt}{\omega t}
\newcommand{\e}{\epsilon}
\renewcommand{\t}{\tau}
\renewcommand{\k}{\kappa}
\newcommand{\g}{\gamma}
\newcommand{\x}{\chi}
\newcommand{\s}{\sigma}


%%Tensors
\newcommand{\arbten}[1]{\tensor{#1}{^{b_1 b_2\cdots b_k}_{c_1c_2\cdots c_l}} }
\newcommand{\arbtenup}[1]{\tensor{#1}{^{b_1 b_2\cdots d\cdots b_k}_{c_1c_2\cdots c_l}} }
\newcommand{\arbtendown}[1]{\tensor{#1}{^{b_1 b_2\cdots b_k}_{c_1c_2\cdots d\cdots c_l}} }




%%Math Text
\newcommand{\Tr}{\mathrm{Tr}}
\newcommand{\?}{\mathcal}
\newcommand{\diag}{\mathrm{diag}}
\newcommand{\sgn}{\mathrm{sgn}}
\newcommand{\sech}{\mathrm{sech}}
\newcommand{\csch}{\mathrm{csch}}
\newcommand{\arctanh}{\mathrm{arctanh}}
\newcommand{\arcsinh}{\mathrm{arcsinh}}
\newcommand{\re}{\mathrm{Re}}
\newcommand{\im}{\mathrm{Im}}
\newcommand{\ID}{\mathbbm{1}}	
\newcommand{\uub}[2]{\underbrace{#1}_{#2}}	

